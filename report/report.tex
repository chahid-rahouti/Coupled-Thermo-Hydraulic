\documentclass[a4paper,12pt]{report}
\usepackage[utf8]{inputenc}
\usepackage{amsmath}
\usepackage{amssymb}
\usepackage{amssymb}
\usepackage{graphicx}
\usepackage{hyperref}
\usepackage{caption}
\usepackage{geometry}
\usepackage{lmodern}
\usepackage{tikz}
\usepackage{tabularx}
\usepackage{hyperref}
\usepackage{mathtools}
% \usepackage{colorlinks=true,linkcolor=blue}
% Définition des marges
\geometry{left=1.5cm, right=2cm, top=1.5cm, bottom=1.5cm}

\newtheorem{theorem}{Theorem}
\newtheorem{definition}{Definition}
\newtheorem{example}{Example}
\newtheorem{remark}{Remark}
\newtheorem{corollary}{Corollary}
\newtheorem{lemma}{Lemma}
\newtheorem{proposition}{Proposition}
\newtheorem{proof}{Proof}

%%%%%%%%%%%%%%%%%%%%%%%%%%%%%%%%%%%%%%%%%%%%%%%%%%%%%%
\begin{document}

% \hspace{-1cm}{\includegraphics[scale=0.3]{Image/logo_ufr_mathinfo.png}} \hfill
% \raisebox{1.2cm}{\begin{minipage}[t]{0.3\textwidth}
%   {\includegraphics[scale=0.45]{Image/udsTransparent.png}} % Remplacez par le chemin de votre autre image
  
%   {\includegraphics[scale=0.45]{Image/Inria-logo.png}}
% \end{minipage}}



\begin{center}
  \raisebox{-1cm}{ % Ajustez la valeur pour déplacer le minipage verticalement
          \begin{minipage}[c]{10cm}
              \centering \bf
              \bf\Large Computational Mechanics and Poromechanics for Induced Seismicity and Control\\
              % UFR of Mathematiques and Informatique\\
              % Departement of Mathematique 
          \end{minipage}
      } 
\end{center}

\vglue5mm
% \centering\vglue3cm\sl
\vspace{1cm}
% \centering
\begin{center}
  \underline{\bf\Large Doctoral Candidate Challenge – ERC INJECT Project} \\
  % \underline{\bf\Large Innovation Modeling (CSMMaster’s in Scientific Computing andI)}
\end{center}
\vglue1mm
\vspace{0cm}
\begin{center}
\vskip1mm
\setlength{\fboxsep}{18pt} % Ajustez l'espace entre le contenu et le bord de la boîte
\setlength{\fboxrule}{2pt} % Ajustez l'épaisseur du bord de la boîte

\fbox{
            \begin{minipage}{\textwidth}
                \centering
                \textbf{\Large COUPLED THERMO–HYDRAULIC ANALYSIS OF A 2-D GEOTHERMAL RESERVOIR}
            \end{minipage}
        }
\end{center}
\vglue5mm
\vskip5mm
\vspace{0.2cm}
\vfill
\begin{minipage}[t]{10cm}
\hspace{0.7cm}{\underline{\bf  Presented by:}\\[3mm]} \bf
CHAHID RAHOUTI\\
\end{minipage}
\hfill
\begin{minipage}[t]{5cm}
\underline{\bf Supervised by:}\\[4mm]\bf
\hfill Dr.L0ANNIS   \\ STEFANOU
\end{minipage}
\vfill
\vfill
% \centering
\begin{center}
  \today
\end{center}

%%%%%%%%%%%%%%%%%%%%%%%%%%%%%%%%%%%%%%%%
% \title{Structure preserving time integration methods for ODEs, e.g. Kepler problem or Hénon-Heiles model, i.e. problems from Astrophysics. }
% \author{Rahouti Chahid, Adama Dieng}
% \date{\today}


% \maketitle
\newpage
\tableofcontents
\newpage
\renewcommand{\thesection}{\arabic{section}} % Redéfinir la numérotation des sections
\setcounter{section}{0} 
\section{Problem description}
An Enhanced Geothermal System (EGS) is being piloted. Two nearly vertical wells, 500 m
apart, intersect a hydraulically stimulated fracture network at a depth of roughly 5 km. Fluids
are injected in one well and produced from the other, establishing a forced-convection loop
that extracts deep subsurface heat.
Although the field layout is three-dimensional, you will model a two-dimensional vertical
cross-section that slices through both wells. This simplification retains the
essential physics while keeping computational cost modest.\\
In the absence of solid matrix deformation, the fluid pressure p(x, t) and temperature T (x, t)
evolve according to a pair of coupled diffusion equations derived from Biot’s
porothermoelasticity:
\begin{align}
&\frac{\partial\, \Delta p}{\partial t} - c_p \nabla^2 \Delta p = \lambda_{pT} \frac{\partial\, \Delta T}{\partial t} \\
    &\frac{\partial\, \Delta T}{\partial t} - c_T \nabla^2 \Delta T = \lambda_{Tp} \frac{\partial\, \Delta p}{\partial t}
\end{align}

\section{Weak formulation and Implementation}
In this section, we will derive the weak formulation of the  this couple of equations
and then given the geometry associated this problem using GMSH software. We will then implement the weak
formulation using the finite element method (FEM) by using a Python implementation.
\subsection{Weak formulation}
We start by giving the system of equations with boundary conditions:
\begin{equation}
\left\{
\begin{aligned}
    &\frac{\partial\, \Delta p}{\partial t} - c_p \nabla^2 \Delta p = \lambda_{pT} \frac{\partial\, \Delta T}{\partial t}, && x \in \Omega,\ t > 0 \\[2mm]
    &\frac{\partial\, \Delta T}{\partial t} - c_T \nabla^2 \Delta T = \lambda_{Tp} \frac{\partial\, \Delta p}{\partial t}, && x \in \Omega,\ t > 0 \\[2mm]
    &\Delta p = 2\,\text{MPa},\quad \Delta T = -130\,^\circ\text{C}, && x \in \Gamma_{\text{inj}},\ t > 0 \\[2mm]
    &\Delta p = -1\,\text{MPa},\quad \mathbf{n} \cdot \nabla \Delta T = 0, && x \in \Gamma_{\text{ext}},\ t > 0 \\[2mm]
    &\mathbf{n} \cdot \nabla \Delta p = 0,\quad \mathbf{n} \cdot \nabla \Delta T = 0, && x \in \Gamma_{\text{bord}},\ t > 0
\end{aligned}
\right.
\end{equation}
where $\Delta p = p - p_0$ and $\Delta T = T - T_0$ are the pressure and temperature variations.
Before starting the weak formulation, we introduce the functional spaces associated with this problem. Due to the mixed boundary conditions, we distinguish two main types of spaces: the solution spaces and the test function spaces.
\begin{equation}
\begin{aligned}
&S_p =  \left\{\, \Delta p \in H^1(\Omega)\ \middle|\ \Delta p = 2\,\text{MPa} \text{ sur } \Gamma_{\text{inj}},\;\; \Delta p = -1\,\text{MPa} \text{ sur } \Gamma_{\text{ext}}\, \right\} \\
&S_t =  \left\{\, \Delta T \in H^1(\Omega)\ \middle|\ \Delta T = -130\,^\circ\text{C} \text{ sur } \Gamma_{\text{inj}}\, \right\}
\end{aligned}
\end{equation}

\begin{equation}
\begin{aligned}
&V_p = \{u \in H^1(\Omega),\ u = 0\ \text{on}\ \Gamma_{\text{inj}} \cup \Gamma_{\text{ext}}\} \\
&V_t = \{v \in H^1(\Omega),\ v = 0\ \text{on}\ \Gamma_{\text{inj}}\}
\end{aligned}
\end{equation}
Then, let $u \in V_p $ and $v \in V_t$ be the test functions, so, by Green's theorem and integration by parts we can write the weak formulation of the problem as follows:
\begin{equation}
\left\{
\begin{aligned}
    &\int_{\Omega} \frac{\partial\, \Delta p}{\partial t} u \,dx + c_p \int_{\Omega} \nabla \Delta p \cdot \nabla u \,dx - c_p \int_{\Gamma_{\text{inj}} \cup \Gamma_{\text{ext}}\cup \Gamma_{\text{bord}}} (\nabla \Delta p \cdot n)  u \,dx   = \lambda_{pT} \int_{\Omega} \frac{\partial\, \Delta T}{\partial t} u \,dx,  && \forall u \in V_p \\[2mm]
    &\int_{\Omega} \frac{\partial\, \Delta T}{\partial t} v \,dx + c_T \int_{\Omega} \nabla \Delta T \cdot \nabla v \,dx - c_T \int_{\Gamma_{\text{inj}} \cup \Gamma_{\text{ext}}\cup \Gamma_{\text{bord}}} (\nabla \Delta T \cdot n)  v \,dx = \lambda_{Tp} \int_{\Omega} \frac{\partial\, \Delta p}{\partial t} v \,dx, && \forall v \in V_t
\end{aligned}
\right.
\end{equation}
After sampling this weak formulation by the bord conditions, we can write the weak formulation as follows:
\begin{equation}
\left\{
\begin{aligned}
    &\int_{\Omega} \frac{\partial\, \Delta p}{\partial t} u \,dx + c_p \int_{\Omega} \nabla \Delta p \cdot \nabla u \,dx  = \lambda_{pT} \int_{\Omega} \frac{\partial\, \Delta T}{\partial t} u \,dx,  && \forall u \in V_p \\[2mm]
    &\int_{\Omega} \frac{\partial\, \Delta T}{\partial t} v \,dx + c_T \int_{\Omega} \nabla \Delta T \cdot \nabla v \,dx  = \lambda_{Tp} \int_{\Omega} \frac{\partial\, \Delta p}{\partial t} v \,dx, && \forall v \in V_t
\end{aligned}
\right.
\end{equation}

These two forms are bilinear (thanks to the linearity of the integral), continuous (due to the continuous embedding of $H^1$ into $L^2$), and coercive (thanks to the Poincaré inequality). Therefore, we can use the Lax-Milgram theorem to prove the existence and uniqueness of the solution to this problem.\\
Find $(\Delta p, \Delta T) \in S_p \times S_t$ such that:
\begin{equation}
\left\{
\begin{aligned}
    &\int_{\Omega} \frac{\partial\, \Delta p}{\partial t} u \,dx + c_p \int_{\Omega} \nabla \Delta p \cdot \nabla u \,dx  = \lambda_{pT} \int_{\Omega} \frac{\partial\, \Delta T}{\partial t} u \,dx,  && \forall u \in V_p \\[2mm]
    &\int_{\Omega} \frac{\partial\, \Delta T}{\partial t} v \,dx + c_T \int_{\Omega} \nabla \Delta T \cdot \nabla v \,dx  = \lambda_{Tp} \int_{\Omega} \frac{\partial\, \Delta p}{\partial t} v \,dx, && \forall v \in V_t
\end{aligned}
\right.
\end{equation}
Now, we giving the matrix form of the weak formulation, we can write the system as follows:
\begin{equation}
\left\{
\begin{aligned}
    & M_p \dot{\Delta p} + K_p \Delta p = \lambda_{pT} M_pt \Delta T \\[2mm]
    & M_t \dot{\Delta T} + K_t \Delta T = \lambda_{Tp} M_p \Delta p
\end{aligned}
\right.
\end{equation}
where $M_p$ and $M_t$ are the mass matrices, and $K_p$ and $K_t$ are the stiffness 
matrices associated with the pressure and temperature equations, respectively. 
The terms $\lambda_{pT}$ and $\lambda_{Tp}$ are coupling coefficients 
between the pressure and temperature equations. I will use Euler backward method and theta method to solve this system of equations.\\
\subsection{Geometry and mesh generation}
The geometry of the problem is a 2D vertical cross-section of the geothermal 
reservoir, which can be represented as a rectangle with two wells. The left well is 
the injection well, and the right well is the production well. 
\begin{figure}[h!]
    \centering
    \begin{minipage}[t]{0.48\textwidth}
        \centering
        \includegraphics[width=\textwidth]{Image/trous.png}
        \caption{Geometry with wells as circles}
        \label{fig:geo_segments}
    \end{minipage}
    \hfill
    \begin{minipage}[t]{0.48\textwidth}
        \centering
        \includegraphics[width=\textwidth]{Image/Line.png}
        \caption{Geometry with wells as segments}
        \label{fig:geo_ronds}
    \end{minipage}
\end{figure}
\subsection{Implementation and Results}
The implementation of this system of equations is done using the finite 
element method (FEM) in Python. To respect the units used in the computations, I performed some conversions, such as from 
Celsius to Kelvin and for the initial condition we have the reservoir is at equilibrium it means that \ref{eq:initial_condition}. You can find the code in the link of the GitHub repository.
\href{https://github.com/chahid-rahouti/Coupled-Thermo-Hydraulic}{https://github.com/chahid-rahouti/Coupled-Thermo-Hydraulic}

\begin{equation}
\Delta p(0,x) = p(0,x) - p_0(0,x) = 0 \quad \text{et} \quad \Delta T(0,x) = T(0,x) - T_0(0,x) = 0
\label{eq:initial_condition}
\end{equation}
\begin{figure}[h!]
    \centering
        \centering
        \includegraphics[width=\textwidth]{Image/solution_couplée.png}
        \caption{Solution of the coupled system of equations for pressure and temperature over time (30 days, 6 mouths, 1 year, 3 years, 10 years)}
        \label{fig:results}
\end{figure}
The results show the evolution of the pressure and temperature in 
the geothermal reservoir over time. The pressure increases in 
the injection well and decreases in the production well, while the temperature 
decreases in the injection well and increases in the production well. 
This behavior is consistent with the expected physics of a geothermal reservoir.(see Figure \ref{fig:results})
I know that the results are not very good, because in the many references i find the distibution
of the pressure and temperature are go to the injection wells to the production wells, but 
in my case the pressure and temperature are not going to the production well. I think there is a problem 
but at this moment I don't know what is the problem. You can see this references (\cite{1}, \cite{2}, \cite{3}, \cite{4}, \cite{5}, \cite{6}).
\section{Conceptual Fault Extension}
In this section, we will extend the conceptual fault model to include the effects of
faults on the pressure and temperature distribution in the geothermal reservoir.
The conceptual fault model is based on the assumption that the faults are
permeable and can allow fluid flow between the reservoir and the fault. 
\subsection{Fault model}
We are modeling the faults as a surface of discontinuity and jump in the pressure and temperature
distribution in the reservoir \ref{fig:fault}. In our case, we can add this surface of discontinuity
at our geometry, and we can use the same weak formulation as before, but we need to add
the effects of the faults. To solve the system in this case, we will use the discontinuous
Galerkin method (DGM). The DGM is a numerical method take into account
the discontinuities in the solution. In this case, we can write the weak formulation of the system with the faults as follows:
\begin{equation}
\left\{
\begin{aligned}
    &\int_{\Omega} \frac{\partial\, \Delta p}{\partial t} u \,dx + c_p \int_{\Omega} \nabla \Delta p \cdot \nabla u \,dx + \int_{\Gamma_{\text{fault}}} \kappa_p [\Delta p] u \,ds = \lambda_{pT} \int_{\Omega} \frac{\partial\, \Delta T}{\partial t} u \,dx,  && \forall u \in V_p \\[2mm]
    &\int_{\Omega} \frac{\partial\, \Delta T}{\partial t} v \,dx + c_T \int_{\Omega} \nabla \Delta T \cdot \nabla v \,dx + \int_{\Gamma_{\text{fault}}} \kappa_T [\Delta T] v \,ds = \lambda_{Tp} \int_{\Omega} \frac{\partial\, \Delta p}{\partial t} v \,dx, && \forall v \in V_t
\end{aligned}
\right.
\end{equation}
where $[\Delta p]$ and $[\Delta T]$ are the jumps in the pressure and temperature across the fault surface $\Gamma_{\text{fault}}$, when $\kappa_p$ and $\kappa_T$ are the fault permeabilities. 
The jumps are defined as the difference between the values of the pressure and temperature on either side of the fault:
\begin{equation}
[\Delta p] = \Delta p^+ - \Delta p^- \quad \text{et} \quad [\Delta T] = \Delta T^+ - \Delta T^-
\end{equation}
\begin{figure}[h!]
    \centering
        \centering
        \includegraphics[width=\textwidth]{Image/fail.png}
        \caption{2D conceptual fault model }
        \label{fig:fault}
\end{figure}
\subsection{Impact of the fault on the reservoir}
% \item Permeable faults (low $\kappa_p$ the fault resistance) can reduce the drawdown by redistributing pressure to zones of lower 
% permeability. It means, approximately a $\sim 30\%$ reduction in 
% the pressure gradient between the injection and production wells.

% \item The fault acts as a preferential pathway for the injected cold 
% water. If $\kappa_T$ is low, the breakthrough time is inversely 
% proportional to $\kappa_T$ (reduction up to 50\%). There is a 
% risk of premature cooling of the production well.

% \textbf{Advantage:} Increased production rate (up to $+20\%$) due to the connection of permeable zones.

% \textbf{Risk:} If $\kappa_p$ is too high, the fault drains the injector towards the exterior, reducing thermal recovery.


\begin{itemize}
    \item \textbf{Pressure redistribution:} Permeable faults (low $\kappa_p$ the fault resistance) can reduce the drawdown by redistributing pressure to zones of lower 
permeability. It means, approximately a $\sim 30\%$ reduction in 
the pressure gradient between the injection and production wells (\cite{7}, \cite{8}).

    \item \textbf{Thermal breakthrough:} The fault acts as a preferential pathway for the injected cold 
water. If $\kappa_T$ is low, the breakthrough time is inversely 
proportional to $t_{\text{breakthrough}} \propto \frac{1}{\kappa_T}$ (reduction up to 50\%). There is a 
risk of premature cooling of the production well \cite{9}.

    \item \textbf{Advantage:} Increased production rate (up to $+20\%$) due to the connection of permeable zones.

    \item \textbf{Risk:} If $\kappa_p$ is too high, the fault may drain the injected fluid towards the exterior, reducing thermal recovery.
\end{itemize}

\section{Closed-Loop Control Challenge}
\subsection{Mathematical problem}
In this situation, we search how we can control the temperature difference between the injection and production wells, 
or exactly maximize the duration when $\Delta T_{\text{ext}}(t) \geq \Delta T_{\min}$ for all $t \in [0,\,10\,\text{ans}]$.
So, the injection flow rate $Q_{\text{inj}}(t)$ (or $\Delta p_{\text{inj}}(t)$) becomes a control variable of the problem. we can use the optimal control theory to solve this problem. We formulate a constrained optimization problem:
\begin{equation}
\min_{Q_{\text{inj}}(t)} \underbrace{\int_0^{10\, \text{years}} L(T_{\text{out}}(t), Q_{\text{inj}}(t))\,dt}_{\text{Total control cost}}
\qquad
\text{such that } T_{\text{out}}(t) \geq T_{\min} \quad \forall t
\end{equation}

subject to
\begin{equation}
    T_{\text{out}}(t) \geq T_{\min} \quad \text{for all} \quad t \in [0,\,10\,\text{years}],
\end{equation}
The function $L(T_{\text{out}}(t), Q_{\text{inj}}(t))$ is generally of the form:
\begin{equation}
L(T_{\text{out}}(t), Q_{\text{inj}}(t)) =
\underbrace{\beta_1 [T_{\min} - T_{\text{out}}(t)]_+^2}_{\text{Penalty if } T_{\text{out}} < T_{\min}}
+ \underbrace{\beta_2 Q_{\text{inj}}^2}_{\text{Pumping consumption}}
\end{equation}
where $[x]_+ = \max(0, x)$.
\subsection{Proposed Strategy}

\textbf{Adaptive control:} Adjust $Q_{\text{inj}}$ in real time using a PI-type control law based on the error $e(t) = T_{\text{out}}(t) - T_{\min}$:
\begin{equation}
Q_{\text{inj}}(t) = K_p\, e(t) + K_i \int_{0}^{t} e(\tau)\, d\tau,
\end{equation}
where $K_p$ and $K_i$ are tuned by prior modeling.

\textbf{Operational limits:} $Q_{\min} \leq Q_{\text{inj}} \leq Q_{\max}$ to avoid fracturing or critical pressure drop.











\addcontentsline{toc}{chapter}{Bibliography}
\begin{thebibliography}{10}
  \bibitem{1}
  Yuting Luo, Juyan Wei, Meilong Fu, Li Fang, and Xudong L.\\
  \textit{Analysis of Enhanced Geothermal System Reservoir Parameters and Fractures on Heat Recovery Efficiency Based on a Single-Phase Conduction Model.}\\
  Article.

  \bibitem{2}
  Wang Jiwei, Ming Chen, Bo Zhang, et al.\\
  \textit{Numerical Simulation of Hydraulic Fracturing Damage Evolution in Geothermal Reservoirs with Natural Fractures Based on THMD Coupling Model.}\\ 
  Conference Paper, June 2022.\\
  DOI: \href{https://doi.org/10.56952/ARMA-2022-0951}{10.56952/ARMA-2022-0951}.\\
  Available at: \url{https://www.researchgate.net/publication/364247123}
    \bibitem{3}
  Yifan Fan, Shikuan Zhang, Yonghui Huang, Zhonghe Pang, and Hongyan Li.\\
  \textit{Determining the Recoverable Geothermal Resources Using a Numerical Thermo-Hydraulic Coupled Modeling in Geothermal Reservoirs.}\\
  Frontiers in Earth Science, 2022.\\
  Available at: \url{https://www.frontiersin.org/articles/10.3389/feart.2022.XXXXXX}
    \bibitem{4}
  Hua-Peng Chen and Musa D. Aliyu.\\
  \textit{Numerical modelling of coupled thermo-hydraulic problems for long-term geothermal reservoir productivity.}\\
  VII International Conference on Computational Methods for Coupled Problems in Science and Engineering (COUPLED PROBLEMS 2017),\\
  M. Papadrakakis, E. Oñate and B. Schrefler (Eds), 2017.\\
  Department of Engineering Science, University of Greenwich, UK.
    \bibitem{5}
  Musa D. Aliyu and Hua-Peng Chen.\\
  \textit{Numerical modelling of coupled hydro-thermal processes of the Soultz heterogeneous geothermal system.}\\
  ECCOMAS Congress 2016, VII European Congress on Computational Methods in Applied Sciences and Engineering,\\
  M. Papadrakakis, V. Papadopoulos, G. Stefanou, V. Plevris (eds.), Crete Island, Greece, 5–10 June 2016.\\
  Department of Engineering Science, University of Greenwich, UK.
    \bibitem{6}
  S. N. Pandey and Vikram Vishal.\\
  \textit{Sensitivity analysis of coupled processes and parameters on the performance of enhanced geothermal systems.}\\
  Scientific Reports, 7:17057, 2017.\\
  DOI: \href{https://doi.org/10.1038/s41598-017-14273-4}{10.1038/s41598-017-14273-4}.\\
  Available at: \url{https://www.nature.com/articles/s41598-017-14273-4}
  \bibitem{7}
    Alexandros Daniilidis, Sanaz Saeid, Nima Gholizadeh Doonechaly.\\
    \textit{The fault plane as the main fluid pathway: Geothermal field development options under subsurface and operational uncertainty.}\\
    Geothermics, Volume 85, 2020, 101771.\\
    Delft University of Technology, University of Geneva, ETH Zurich.\\
    DOI: \href{https://doi.org/10.1016/j.geothermics.2019.101771}{10.1016/j.geothermics.2019.101771}.\\
    Available at: \url{https://www.sciencedirect.com/science/article/pii/S0375650519303372}
    \bibitem{8}
  Kingsley Anyim, Quan Gan.\\
  \textit{Fault zone exploitation in geothermal reservoirs: Production optimization, permeability evolution and induced seismicity.}\\
  Geothermics, Volume 111, 2024, 102899.\\
  DOI: \href{https://doi.org/10.1016/j.geothermics.2023.102899}{10.1016/j.geothermics.2023.102899}.\\
  Available at: \url{https://www.sciencedirect.com/science/article/pii/S0375650523001842}
    \bibitem{9}
  Caroline Zaal, Alexandros Daniilidis, Femke C. Vossepoel.\\
  \textit{Economic and fault stability analysis of geothermal field development in direct-use hydrothermal reservoirs.}\\
  Geothermal Energy, 2022.\\
  DOI: \href{https://doi.org/10.1186/s40517-022-00237-5}{10.1186/s40517-022-00237-5}.\\
  Available at: \url{https://geothermal-energy-journal.springeropen.com/articles/10.1186/s40517-022-00237-5}
  \bibitem{10}
  Charbel Salameh, Patrick Schalbart, Bruno Peuportier.\\
  \textit{Optimal Control Strategies for Energy Production Systems using Buildings Thermal Mass.}
    \bibitem{11}
  Zhihong Lei, Yulong Zhang, Qiliang Cui, Yu Shi.\\
  \textit{The injection‑production performance of an enhanced geothermal system considering fracture network complexity and thermo‑hydro‑mechanical coupling in numerical simulations.}\\
  Scientific Reports, 2023.\\
  Available at: \url{https://www.nature.com/articles/s41598-023-31913-6}
  
\end{thebibliography}

\end{document}
